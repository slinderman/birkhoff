\documentclass{article}

\RequirePackage[l2tabu, orthodox]{nag}
\documentclass{article}

% FONTS
\usepackage[T1]{fontenc}

% Replace default Latin Modern typewriter with its proportional counterpart
% http://www.tug.dk/FontCatalogue/lmoderntypewriterprop/
\renewcommand*\ttdefault{lmvtt}


%%% OPTION 1 - Fourier Math + New Century Schoolbook + ParaType Sans

% % Import Fourier Math (this imposes its own New Century Schoolbook type)
% % http://www.ctan.org/tex-archive/fonts/fouriernc/
%\usepackage{fouriernc}
%\usepackage{amsmath}
% % Replace with TeX Gyre Schola version of New Century Schoolbook (must scale!)
% % http://www.tug.dk/FontCatalogue/tgschola/
%\usepackage[scale=0.92]{tgschola}
%\usepackage[scaled=0.88]{PTSans}

%% OPTION 2 - MathDesign Math + Bitstream Charter + ParaType Sans

% Import MathDesign (this brings along Bitstream Charter)
% http://www.ctan.org/tex-archive/fonts/mathdesign/
\usepackage[bitstream-charter]{mathdesign}
\usepackage{amsmath}
\usepackage[scaled=0.92]{PTSans}


% %%% OPTION 3 - MTPRO 2 Math + Termes Times + ParaType Sans

% \usepackage{tgtermes}
% \usepackage{amsmath}
% \usepackage[subscriptcorrection,
%             amssymbols,
%             mtpbb,
%             mtpcal,
%             nofontinfo  % suppresses all warnings
%            ]{mtpro2}
% \usepackage{scalefnt,letltxmacro}
% \LetLtxMacro{\oldtextsc}{\textsc}
% \renewcommand{\textsc}[1]{\oldtextsc{\scalefont{1.10}#1}}
% \usepackage[scaled=0.92]{PTSans}

% GEOMETRY
\usepackage[
  paper  = letterpaper,
  left   = 1.65in,
  right  = 1.65in,
  top    = 1.0in,
  bottom = 1.0in,
  ]{geometry}

% COLOR
\usepackage[usenames,dvipsnames]{xcolor}
\definecolor{shadecolor}{gray}{0.9}

% SPACING and TEXT
\usepackage[final,expansion=alltext]{microtype}
\usepackage[english]{babel}
\usepackage[parfill]{parskip}
\usepackage{afterpage}
\usepackage{framed}
\usepackage{verbatim}

%redefine the leftbar environment to accept a width and coloring options
\renewenvironment{leftbar}[1][\hsize]
{%
  \def\FrameCommand
  {%
    {\color{Gray}\vrule width 3pt}%
    \hspace{10pt}%
    %\hspace{0pt}\fboxsep=\FrameSep\colorbox{black!10}%
  }%
  \MakeFramed{\hsize#1\advance\hsize-\width\FrameRestore}%
}%
{\endMakeFramed}

% define a paragraph header function
\DeclareRobustCommand{\parhead}[1]{\textbf{#1}~}

% EDITING
% line numbering in left margin
\usepackage{lineno}
\renewcommand\linenumberfont{\normalfont
                             \footnotesize
                             \sffamily
                             \color{SkyBlue}}
% ragged paragraphs in right margin
\usepackage{ragged2e}
\DeclareRobustCommand{\sidenote}[1]{\marginpar{
                                    \RaggedRight
                                    \textcolor{Plum}{\textsf{#1}}}}
% paragraph counter in right margin
\newcommand{\parnum}{\bfseries\P\arabic{parcount}}
\newcounter{parcount}
\newcommand\p{%
    \stepcounter{parcount}%
    \leavevmode\marginpar[\hfill\parnum]{\parnum}%
}
% paragraph helper
%\DeclareRobustCommand{\PP}{\textcolor{Plum}{\P} }

% COUNTERS
\renewcommand{\labelenumi}{\color{black!67}{\arabic{enumi}.}}
\renewcommand{\labelenumii}{{\color{black!67}(\alph{enumii})}}
\renewcommand{\labelitemi}{{\color{black!67}\textbullet}}

% FIGURES
\usepackage{graphicx}
\usepackage[labelfont=bf]{caption}
\usepackage[format=hang]{subcaption}

% TABLES
\usepackage{booktabs}

% ALGORITHMS
\usepackage[algoruled]{algorithm2e}
\usepackage{listings}
\usepackage{fancyvrb}
\fvset{fontsize=\normalsize}

% BIBLIOGRAPHY
\usepackage{natbib}

% HYPERREF
\usepackage[colorlinks,linktoc=all]{hyperref}
\usepackage[all]{hypcap}
\hypersetup{citecolor=BurntOrange}
\hypersetup{linkcolor=MidnightBlue}
\hypersetup{urlcolor=MidnightBlue}

% CLEVEREF must come after HYPERREF
\usepackage[nameinlink]{cleveref}

% ACRONYMS
\usepackage[acronym,smallcaps,nowarn]{glossaries}
% \makeglossaries

% COLOR DEFINITIONS
\newcommand{\red}[1]{\textcolor{BrickRed}{#1}}
\newcommand{\orange}[1]{\textcolor{BurntOrange}{#1}}
\newcommand{\green}[1]{\textcolor{OliveGreen}{#1}}
\newcommand{\blue}[1]{\textcolor{MidnightBlue}{#1}}
\newcommand{\gray}[1]{\textcolor{black!60}{#1}}

% LISTINGS DEFINTIONS
\lstdefinestyle{mystyle}{
    commentstyle=\color{OliveGreen},
    keywordstyle=\color{BurntOrange},
    numberstyle=\tiny\color{black!60},
    stringstyle=\color{MidnightBlue},
    basicstyle=\ttfamily,
    breakatwhitespace=false,
    breaklines=true,
    captionpos=b,
    keepspaces=true,
    numbers=left,
    numbersep=5pt,
    showspaces=false,
    showstringspaces=false,
    showtabs=false,
    tabsize=2
}
\lstset{style=mystyle}

\DeclareRobustCommand{\mb}[1]{\ensuremath{\boldsymbol{\mathbf{#1}}}}
\DeclareRobustCommand{\KL}[2]{\ensuremath{\textrm{KL}\left(#1\;\|\;#2\right)}}

\newcommand{\supp}{\textrm{supp}}

\newcommand{\E}{\mathbb{E}}
\newcommand{\Var}{\mathbb{V}\textrm{ar}}

% Redundant with reals, naturals, below
\newcommand{\bbN}{\mathbb{N}}
\newcommand{\bbZ}{\mathbb{Z}}
\newcommand{\bbR}{\mathbb{R}}
\newcommand{\bbS}{\mathbb{S}}
\newcommand{\bbH}{\mathbb{H}}


\newcommand{\bA}{\boldsymbol{A}}
\newcommand{\bB}{\boldsymbol{B}}
\newcommand{\bC}{\boldsymbol{C}}
\newcommand{\bD}{\boldsymbol{D}}
\newcommand{\bE}{\boldsymbol{E}}
\newcommand{\bF}{\boldsymbol{F}}
\newcommand{\bG}{\boldsymbol{G}}
\newcommand{\bH}{\boldsymbol{H}}
\newcommand{\bI}{\boldsymbol{I}}
\newcommand{\bJ}{\boldsymbol{J}}
\newcommand{\bK}{\boldsymbol{K}}
\newcommand{\bL}{\boldsymbol{L}}
\newcommand{\bM}{\boldsymbol{M}}
\newcommand{\bN}{\boldsymbol{N}}
\newcommand{\bO}{\boldsymbol{O}}
\newcommand{\bP}{\boldsymbol{P}}
\newcommand{\bQ}{\boldsymbol{Q}}
\newcommand{\bR}{\boldsymbol{R}}
\newcommand{\bS}{\boldsymbol{S}}
\newcommand{\bT}{\boldsymbol{T}}
\newcommand{\bU}{\boldsymbol{U}}
\newcommand{\bV}{\boldsymbol{V}}
\newcommand{\bW}{\boldsymbol{W}}
\newcommand{\bX}{\boldsymbol{X}}
\newcommand{\bY}{\boldsymbol{Y}}
\newcommand{\bZ}{\boldsymbol{Z}}
\newcommand{\ba}{\boldsymbol{a}}
\newcommand{\bb}{\boldsymbol{b}}
\newcommand{\bc}{\boldsymbol{c}}
\newcommand{\bd}{\boldsymbol{d}}
\newcommand{\be}{\boldsymbol{e}}
\newcommand{\bbf}{\boldsymbol{f}}
\newcommand{\bg}{\boldsymbol{g}}
\newcommand{\bh}{\boldsymbol{h}}
\newcommand{\bi}{\boldsymbol{i}}
\newcommand{\bj}{\boldsymbol{j}}
\newcommand{\bk}{\boldsymbol{k}}
\newcommand{\bl}{\boldsymbol{l}}
\newcommand{\bbm}{\boldsymbol{m}}
\newcommand{\bn}{\boldsymbol{n}}
\newcommand{\bo}{\boldsymbol{o}}
\newcommand{\bp}{\boldsymbol{p}}
\newcommand{\bq}{\boldsymbol{q}}
\newcommand{\br}{\boldsymbol{r}}
\newcommand{\bs}{\boldsymbol{s}}
\newcommand{\bt}{\boldsymbol{t}}
\newcommand{\bu}{\boldsymbol{u}}
\newcommand{\bv}{\boldsymbol{v}}
\newcommand{\bw}{\boldsymbol{w}}
\newcommand{\bx}{\boldsymbol{x}}
\newcommand{\by}{\boldsymbol{y}}
\newcommand{\bz}{\boldsymbol{z}}

\newcommand{\balpha}{\boldsymbol{\alpha}}
\newcommand{\bbeta}{\boldsymbol{\beta}}
\newcommand{\boldeta}{\boldsymbol{\eta}}
\newcommand{\bkappa}{\boldsymbol{\kappa}}
\newcommand{\bgamma}{\boldsymbol{\gamma}}
\newcommand{\blambda}{\boldsymbol{\lambda}}
\newcommand{\bmu}{\boldsymbol{\mu}}
\newcommand{\bnu}{\boldsymbol{\nu}}
\newcommand{\brho}{\boldsymbol{\rho}}
\newcommand{\bphi}{\boldsymbol{\phi}}
\newcommand{\bpi}{\boldsymbol{\pi}}
\newcommand{\bpsi}{\boldsymbol{\psi}}
\newcommand{\bsigma}{\boldsymbol{\sigma}}
\newcommand{\btheta}{\boldsymbol{\theta}}
\newcommand{\bomega}{\boldsymbol{\omega}}
\newcommand{\bxi}{\boldsymbol{\xi}}
\newcommand{\bGamma}{\boldsymbol{\Gamma}}
\newcommand{\bLambda}{\boldsymbol{\Lambda}}
\newcommand{\bOmega}{\boldsymbol{\Omega}}
\newcommand{\bPhi}{\boldsymbol{\Phi}}
\newcommand{\bPi}{\boldsymbol{\Pi}}
\newcommand{\bPsi}{\boldsymbol{\Psi}}
\newcommand{\bSigma}{\boldsymbol{\Sigma}}
\newcommand{\bTheta}{\boldsymbol{\Theta}}
\newcommand{\bUpsilon}{\boldsymbol{\Upsilon}}
\newcommand{\bXi}{\boldsymbol{\Xi}}
\newcommand{\bepsilon}{\boldsymbol{\epsilon}}

\newcommand{\mcA}{\mathcal{A}}
\newcommand{\mcB}{\mathcal{B}}
\newcommand{\mcC}{\mathcal{C}}
\newcommand{\mcD}{\mathcal{D}}
\newcommand{\mcE}{\mathcal{E}}
\newcommand{\mcF}{\mathcal{F}}
\newcommand{\mcG}{\mathcal{G}}
\newcommand{\mcH}{\mathcal{H}}
\newcommand{\mcI}{\mathcal{I}}
\newcommand{\mcJ}{\mathcal{J}}
\newcommand{\mcK}{\mathcal{K}}
\newcommand{\mcL}{\mathcal{L}}
\newcommand{\mcM}{\mathcal{M}}
\newcommand{\mcN}{\mathcal{N}}
\newcommand{\mcO}{\mathcal{O}}
\newcommand{\mcP}{\mathcal{P}}
\newcommand{\mcQ}{\mathcal{Q}}
\newcommand{\mcR}{\mathcal{R}}
\newcommand{\mcS}{\mathcal{S}}
\newcommand{\mcT}{\mathcal{T}}
\newcommand{\mcU}{\mathcal{U}}
\newcommand{\mcV}{\mathcal{V}}
\newcommand{\mcW}{\mathcal{W}}
\newcommand{\mcX}{\mathcal{X}}
\newcommand{\mcY}{\mathcal{Y}}
\newcommand{\mcZ}{\mathcal{Z}}

\newcommand{\trans}{\mathsf{T}}
\newcommand{\naturals}{\mathbb{N}}
\newcommand{\reals}{\mathbb{R}}
\def\argmax{\operatornamewithlimits{arg\,max}}
\def\argmin{\operatornamewithlimits{arg\,min}}

\newcommand{\distNormal}{\mathcal{N}}
\newcommand{\distGamma}{\mathrm{Gamma}}
\newcommand{\distBernoulli}{\mathrm{Bern}}
\newcommand{\distBinomial}{\mathrm{Bin}}
\newcommand{\distCategorical}{\mathrm{Cat}}
\newcommand{\distDirichlet}{\mathrm{Dir}}
\newcommand{\distMultinomial}{\mathrm{Mult}}
\newcommand{\distPolyaGamma}{\mathrm{PG}}
\newcommand{\distMNIW}{\mathrm{MNIW}}
\newcommand{\distBeta}{\mathrm{Beta}}

\newcommand{\prt}[1]{\frac{\partial}{\partial #1}}
\newcommand{\deriv}[1]{\frac{\mathrm{d}}{\mathrm{d} #1}}


\newcommand{\TODO}[1]{\textcolor{red}{[TODO: #1]}}

\newcommand{\bbI}{\mathbb{I}}
\newcommand{\bbE}{\mathbb{E}}
\newcommand{\bone}{\boldsymbol{1}}
\newcommand{\bigO}{\mathcal{O}}
\newcommand{\iid}[1]{\stackrel{\text{iid}}{#1}}
\newcommand\indep{\protect\mathpalette{\protect\independenT}{\perp}}
\def\independenT#1#2{\mathrel{\rlap{$#1#2$}\mkern4mu{#1#2}}}
\DeclareMathOperator{\Skew}{Skew}
\DeclareMathOperator{\Symm}{Sym}
\DeclareMathOperator{\tr}{tr}

%\DeclareMathOperator{\KL}{KL}
\newcommand{\given}{\, | \,}

\DeclareMathOperator{\diag}{diag}
\let\vec\relax% Set equal to \relax so that LaTeX thinks it's not defined
\DeclareMathOperator{\vec}{vec}
\let\Re\relax
\DeclareMathOperator{\Re}{\textup{Re}}
\let\Im\relax
\DeclareMathOperator{\Im}{\textup{Im}}

% Backcompat: dif and diff both work
\newcommand*\dif{\mathop{}\!\mathrm{d}}
\newcommand*\diff{\mathop{}\!\mathrm{d}}


% !TEX root = template.tex

\newacronym{KL}{kl}{Kullback-Leibler}
\newacronym{ELBO}{elbo}{evidence lower bound}
\newacronym{SVI}{svi}{stochastic variational inference}
\newacronym{GMM}{gmm}{Gaussian mixture model}
\newacronym{LDA}{lda}{latent Dirichlet allocation}



\usepackage{blindtext}

\usepackage{icml2016}
% \usepackage[accepted]{icml2016}

% The \icmltitle you define below is probably too long as a header.
% Therefore, a short form for the running title is supplied here:
\icmltitlerunning{Short Title}

\begin{document}

\twocolumn[
\icmltitle{Reparameterizing the Birkhoff Polytope for Variational Permutation Inference}

\icmlauthor{Your Name}{email@yourdomain.edu}
\icmladdress{Your Fantastic Institute,
            314159 Pi St., Palo Alto, CA 94306 USA}
\icmlauthor{Your CoAuthor's Name}{email@coauthordomain.edu}
\icmladdress{Their Fantastic Institute,
            27182 Exp St., Toronto, ON M6H 2T1 CANADA}

\vskip 0.3in
]

\begin{abstract}
How to perform inference over the space of permutation matrices?  By
definition, with~$n$ nodes, there are~$n!$ such matrices.  Clearly,
searching or optimizing over this space quickly becomes intractable
as~$n$ grows. Our goal is to derive a tractable algorithm for
performing approximate inference over this challenging discrete space.
To that end, we
propose a continuous relaxation of permutation matrices to doubly
stochastic matrices, i.e. to points in the Birkhoff polytope.  We then
derive an invertible and differentiable map from densities on
unconstrained space to densities on the Birkhoff polytope. This
transformation is parameterized by a ``temperature'' that controls how
concentrated the resulting density is at the extrema of the Birkhoff
polytope; i.e. at permutation matrices.  This relaxation admits
variational inference via stochastic gradient ascent over the
distributions on doubly stochastic matrices (and in the zero-temperature
limit, on permutation matrices) using Monte Carlo estimates of the
reparameterized gradient.
\end{abstract}

\section{Continuous relaxations for discrete inference}

In Bayesian inference problems, we have a prior distribution~$p(x)$
and a likelihood~$p(y \mid x)$, and
we seek the posterior distribution,~${p(x \mid y) = p(x) p(y \mid x) / p(y)}$.
%The output a distribution over vertices of the simplex.
In general, this problem is intractable since the normalizing constant
in Bayes' rule, $p(y)$, involves a high dimensional integral or sum.
Variational inference algorithms avoid this problem by limiting their
search to a tractable family of distributions,~$q(x; \theta)$,
parameterized by~$\theta$, and searching for the member of this family
that best approximates the true posterior. Most commonly, the
approximation quality is measured by the Kullback-Leibler (KL)
divergence between the variational posterior,~$q(x; \theta)$, and the
true posterior,~$p(x \mid y)$. That is, the optimal variational
parameters are given by,
\begin{align}
  \theta^* &= \argmax_\theta -\KL{q(x; \theta)}{p(x \mid y)},
\end{align}
where
\begin{align}
  -\KL{q(x; \theta)}{p(x \mid y)} 
  &= \E_{q}
  \left[ \log p(x \mid y) - \log q(x; \theta) \right] \\
  &\geq \E_{q}
  \left[ \log p(x, \, y) - \log q(x; \theta) \right] \\
  &= \cL(\theta).
\end{align}
The objective function,~$\cL(\theta)$, is known as the evidence lower bound, or ELBO.
Stochastic gradient ascent is
perhaps the simplest method of optimizing the ELBO with respect to
the parameters~$\theta$.
However, computing~$\nabla_\theta \cL(\theta)$ requires some care,
since the ELBO contains an expectation with respect to a distribution
that depends on these parameters.

When~$x$ is a continuous random variable, we can often go one step
further and leverage the ``reparameterization trick''  \citep{Salimans2013,Kingma2014,Price1958,Bonnet1964}.  Specifically,
in some cases we can simulate from~$q$ via the following procedure,
\begin{align}
\xi &\sim p(\xi),  \\
x &= g(\theta, \xi),
\end{align}
where~$g(\theta, \xi)$ is a deterministic and differentiable
transformation (which is only possible if~$x$ is continuous). Thus, we
can effectively ``factor out'' the randomness of~$q$. With this
transformation, we can bring the gradient inside the expectation as
follows,
\begin{align}
%  \nabla_\theta \E_{q(x \mid \theta)} \left[ \log p(x \mid y) - \log q(x; \theta) \right]
  \nabla_\theta \cL(\theta) 
  &= \E_{p(\xi)} \left[ \nabla_\theta \left[\log p(g(\theta, \xi) \mid y) - \log q(g(\theta, \xi); \theta) \right] \right].
  %\\
%  &\approx \frac{1}{M} \sum_{m=1}^M \nabla_\theta \log p(g(\theta, \xi^{(m)}) \mid y) - \log q(g(\theta, \xi^{(m)}); \theta) & & & \xi^{(m)} &\sim p(\xi).
\end{align}
This gradient can be estimated with Monte Carlo, and, in practice,
this leads to lower variance estimates of the gradient than, for
example, the score function estimator \citep{Williams1992,Glynn1990}.




Recently, there have been a number of proposals for extending these
reparameterization tricks to high dimensional discrete
problems\footnote{Discrete inference is only problematic in the high
dimensional case, since in low dimensional problems we can enumerate
the possible values of~$x$ and compute the normalizing constant~$p(y)
= \sum_x p(y, x)$.} by relaxing them to analogous continuous
problems \citep{maddison2016concrete, jang2016categorical,
kusner2016gans}.  These approaches are based on the following
observation: if~$x \in \{0,1\}^k$ is a one-hot vector drawn from a
categorical distribution, then the support of~$p(x)$ is the set of
vertices of the~$k-1$ dimensional simplex.  Thus, we can represent the
distribution of~$x$ as an atomic density on the simplex.  Sampling~$x$
is equivalent to sampling this atomic measure. That is,
\begin{align}
  x &\sim \distCategorical( x \mid \theta) &\iff& & {\pi} &\sim p({\pi} \mid \theta) \nonumber \\
  & & & & x &\triangleq {\pi},
\end{align}
where~$p({\pi} \mid \theta)$ is a density on the simplex with atoms at
the~$k$ vertices.

Viewing~$x$ as a vertex of the simplex motivates a natural relaxation:
let us set~$x={\pi}$ as above, but 
rather than restricting~$p({\pi} \mid \theta)$ to be an atomic measure,
let it be a continuous density on the simplex. To be concrete, suppose
the density of~${\pi}$ is defined by the transformation,
\begin{align}
  \xi &\sim p(\xi), \\
  {\pi} &= g(\theta, \xi) \\
  g(\theta, \xi) &= \text{softmax}(\log \theta + \xi).
\end{align}
The output~$x$ is now a point on the simplex, and the parameters~$\theta$ can
be optimized via stochastic gradient ascent with the reparameterization trick,
as discussed above.

In the aforementioned papers,~$p(\xi)$ is taken to be the Gumbel distribution.
This choice leads to a nicely interpretable model: adding
Gumbel noise and taking the argmax yields an exact sample from~$\theta$;
setting~$g$ to the softmax is a natural relaxation. Ultimately, however, this
is just a continuous relaxation of an atomic density to a continuous
density. 

\section{An alternative continuous relaxation}
While the Gumbel-softmax has some nice properties, as we will see, it does
not lend itself as naturally to more complicated generalizations. 
Consider the following alternative model for~${\pi}$:
\begin{align}
  \psi_k &\sim \distNormal(\mu_k, \tau^{-1} \eta_k^2) & &  \text{for } k=1, \ldots, K-1\\
  {\pi}_1 &= \sigma(\psi_1) \\
%  {\pi}_k &= \sigma(\psi_k) \prod_{j=1}^{k-1} (1-\sigma(\psi_j)) & &  \text{for } k=2, \ldots, K-1\\
  {\pi}_k &= \sigma(\psi_k) (1- \sum_{j=1}^{k-1} {\pi}_j) & &  \text{for } k=2, \ldots, K-1\\
%  {\pi}_K &= \prod_{j=1}^{K-1} (1-\sigma(\psi_j))
{\pi}_K &= 1- \sum_{j=1}^{K-1} {\pi}_j
\end{align}
This is known as a logistic stick breaking transformation since
~$\sigma(\cdot)$ is the logistic function and~$\sigma(\psi_k)$ can be
seen as the fraction of the remaining ``stick'' of probability mass
assigned to~${\pi}_k$ \citep{linderman2015dependent}. Moreover,
the density of~${\pi}$ can be expressed in closed form as a
function of~$\mu_k$ and~$\eta_k^2$.  Finally, as with the relaxations
above, the temperature~$\tau$ controls how
concentrated~$p_\tau({\pi} \mid \{\mu_k, \eta_k^2\})$ is at the
vertices of the simplex. As~$\tau \to 0$, the density becomes
concentrated on atoms at the~$K$ vertices, and as~$\tau \to \infty$,
the density concentrates on a point in the interior of the simplex
determined by~$\{\mu_k\}$. For intermediate values, the density is
continuous on the simplex.

\subsection{Limit analysis}
One nice property of the concrete distribution \citep{maddison2016concrete, jang2016categorical} is that it can be understood as the 'heating' of the original categorical distribution (equivalently, a probability simplex-valued distribution whose atoms are the vertices of the simplex): specifically, in the high temperature limit ($\tau \rightarrow \infty$) the distribution becomes a single atom in the center of mass of the probability simplex, while the zero-temperature limit ($\tau\rightarrow 0$) corresponds to the target categorical distribution. Now we show that although the stick-breaking representation may seem 'position biased', in reality, through appropriate choices of the parameters we are able to provide richer representations as compared to the concrete case. Specifically, we will show that degenerate cases can correspond to arbitrary categorical distributions and arbitrary point masses in the simplex.

To see this, consider first the more general stick-breaking representation
\begin{align}
  \psi_k & = g(\theta_k,\epsilon_k)& &  \text{for } k=1, \ldots, K-1\\
  {\pi}_1 &= \psi_1 \\
  {\pi}_k &= \psi_k \left(1- \sum_{j=1}^{k-1} {\pi}_j\right) & &  \text{for } k=2, \ldots, K-1\\
%  {\pi}_K &= \prod_{j=1}^{K-1} (1-\sigma(\psi_j))
{\pi}_K &= 1- \sum_{j=1}^{K-1} {\pi}_j.
\end{align}

Where $g$ is a $[0,1]$-valued function. 

The following lemmas analyze two degenerate but useful cases:

\textbf{Proposition:}  the degenerate case where $\psi_k=g(\theta_k)$ (i.e, $\psi_k$ is non-random) leads to $\pi\sim \delta(\tilde{\pi})$  (i.e, single atom in the point $\tilde{\pi}$) with $\pi$ in $\Delta^{k-1}$ is computed from the above formulae. Also, if any point in $[0,1]$ can be realized through $g(\theta_k)$ then any deterministic $\pi$ can be realized.

\textit{Proof}: The first part is obvious. The second part is also obvious, once acknowledging the invertibility of the function $f$ that maps $\psi \xrightarrow{f} \pi$.

\textbf{Proposition:} the degenerate case where  $\psi_k$ are Bernoulli with some parameter $p_{\theta_k}$) leads to $\pi$ having an atomic distribution with atoms in the vertices of $\Delta^{k-1}$ (i.e, $\pi$ is categorical). We have the following expression for the probabilities of the atoms $\pi_k=1$ (one hot vectors):
\begin{align}
P(\pi_k =1)&= \prod_{i=1}^{k-1} (1-p_{\theta_i}) p_{\theta_k}  & &\text{for } k=1, \ldots, K-1,\\
P(\pi_K =1) &= \prod_{i=1}^{K-1} (1-p_{\theta_i}).
\end{align}
Moreover, if any Bernoulli variable can be realized through appropriate choice of the parameters then, any categorical distribution can be realized.
\textit{Proof}: the first expression comes from expressing the event $\pi_k=1$ equivalently as $\pi_k=1,p_i=0 i<k$ and then, conditioning backwards successively. The second comes from the following (which is deduced by replacing terms in the above)
$$ p_{\theta_k}=\frac{P(\pi_k =1)}{P(\pi_{k-1} =1)}\frac{p_{\theta_{k-1}}}{1-p_{\theta_{k-1}}},\quad k =1,\ldots, K-1.$$
The recursive nature of the above equation gives a recipe to iteratively determine the required $p_{\theta_k}$, given  $P(\pi_k =1), P(\pi_{k-1} =1)$ and the already computed $p_{\theta_{k-1}}$.

Having stated the above properties it only remains showing that the two re-parameterizations we consider here, based on the Gaussian and Kumaraswamy distributions, lead to the above degenerate distribution as their limits.

\textbf{Proposition} For the choice $\psi=\sigma(\delta),\delta\sim\mathcal{N}(\mu,\eta^2)$, the limit $\eta\rightarrow 0$ leads to the non-random $\psi=\sigma(\mu)$. Additionally, the limit $\mu\rightarrow \infty, \eta^2=\mu/K$ with K constant leads to $\psi\sim \text{Bernoulli}(\Phi(K))$ ($\Phi$ denotes the standard normal cdf). In both cases the convergence is in distribution \textit{Proof}. The first convergence is obvious. To see the second, let's index $\mu_n$ and  study the cdf of $\Psi_n$ and 
\begin{align}F_{\Psi_n}(x)&=& P(\sigma(\delta_n)<x) \\
&=&P(\delta_n< \sigma^{-1}(x))\\
&=& P(\mu_n +\mu_n/K\epsilon <\sigma^{-1}(x)),\\\
&=& P( \epsilon <\sigma^{-1}(x))K/\mu_n -K)\\
&=& \Phi( \sigma^{-1}(x))K/\mu_n -K) 
\end{align}
Therefore, by continuity of $\Phi$ we obtain $F_{\Psi_n}(x)\rightarrow \Phi(-K)$ for all points $x\in(,1)$. On the other hand, the cdf of a bernoulli random $F$ variable is given by  a step function that abruptly changes at zero from (zero to $1-p$), and at one (from $1-p$ to 1. As we have obtained convergence to $F$ at all ints continuity points (the interval $(0,1)$), with $1-p= \Phi(-K)\rightarrow \Phi(K)=p$) so we can conclude. Notice that the above representation only allows   to converge to $p>0.5$, as $K$ has to be positive. This can be fixed by choosing sequence with negative $\mu$ instead.

\textbf{Proposition} For the choice $\psi=\mathcal{K}(a,b)$: i) in the limit $a,b \rightarrow $  we converge to a non-random $p$, provided that $p=bB\left(1+\frac{1}{a},b\right)$ along the limiting sequence. ii) In the limit $a,b\rightarrow 0$ we obtain convergence to a Bernoulli random variable with parameter $p$, provided the same condition involving $p,a,b$ holds. In both cases convergence is in probability.


\textit{Proof}: A proof can be found in \cite{mitnik2013kumar}
\section{Continuous relaxations for permutation matrices}
Just as one-hot vectors are the vertices of the simplex, the Birkhoff-von Neumann
theorem states that permutation matrices are vertices of the convex
hull of doubly stochastic matrices.
For permutations of size~$n$, a permutation matrix,~$X \in \{0,1\}^{n \times n}$,
is a binary matrix such that every row and every column sums to one.
An analogous relaxation to the one above is to consider~$X \approx {\Pi}$,
where~${\Pi} \in [0,1]^{n \times n}$ is a doubly stochastic matrix,
i.e. the rows and columns both sum to one. This set is known as the Birkhoff
polytope, which we denote by~$\cB_n$. Due to these constraints, the Birkhoff
polytope lies within a~$(n-1)^2$ dimensional subspace of all $[0,1]^{n \times n}$ matrices. 

We now derive an invertible and differentiable transformation,~$f: \reals^{(n-1) \times (n-1)} \to \cB_n$,
which can be used to define a density on~$\cB_n$. Our approach is an
extension of the stick-breaking transformation described above, with minor
modifications to accomodate the additional constraints of doubly stochastic
matrices. Imagine transforming a real-valued matrix~$\Psi \in \reals^{(n-1) \times (n-1)}$
into a doubly stochastic matrix,~${\Pi} \in [0,1]^{n \times n}$.
We work entry by entry, starting in the top left
and raster scanning left to right then top to bottom. Denote the~$(i,j)$-th entries
of~$\Psi$ and~${\Pi}$ by~$\psi_{ij}$ and~${\pi}_{ij}$, respectively.

The first entry is given by, ${\pi}_{11} = \sigma(\psi_{11})$.
As we work left to right in the first row, the ``remaining stick'' length
decreases as we add new entries. This reflects the row normalization constraints.
Thus,
\begin{align}
%  {\pi}_{1j} &= \sigma(\psi_{1j}) \prod_{k=1}^{j-1} (1-\sigma(\psi_{1k})) & &  \text{for } j=2, \ldots, n-1\\
  {\pi}_{1j} &= \sigma(\psi_{1j}) (1 - \sum_{k=1}^{j-1} {\pi}_{1k})  & &  \text{for } j=2, \ldots, n-1\\
  {\pi}_{1n} &= 1 - \sum_{k=1}^{n-1} {\pi}_{1k}
\end{align}
So far, this is exactly as above. However, the remaining rows must now
conform to both row- and column-constraints. That is,
\begin{align}
{\pi}_{ij} &\leq 1- \sum_{k=1}^{j-1} {\pi}_{ik} & & \text{(row sum)} \\
{\pi}_{ij} &\leq 1- \sum_{k=1}^{i-1} {\pi}_{kj} & & \text{(column sum)}.
\end{align}
Moreover, there is also a lower bound on~${\pi}_{ij}$. This entry must
claim enough of the stick such that what is leftover ``fits'' within the confines
imposed by subsequent column sums. That is, each column sum places an upper
bound on the amount that may be attributed to any subsequent entry. If the remaining
stick exceeds the sum of these upper bounds, the matrix will not be doubly stochastic.
Thus,
\begin{align}
\underbrace{1 - \sum_{k=1}^j \pi_{ik}}_{\text{remaining stick}}
&\leq \underbrace{\sum_{m=j+1}^n (1- \sum_{k=1}^{i-1} {\pi}_{km})}_{\text{remaining upper bounds}}.
\end{align}
Rearranging terms, we have,
\begin{align}
\pi_{ij} &\geq 1- \sum_{k=1}^{j-1} \pi_{ik} - \sum_{m=j+1}^n (1- \sum_{k=1}^{i-1} {\pi}_{km}) \\
&= 1 - n + j - \sum_{k=1}^{j-1} \pi_{ik}  +  \sum_{k=1}^{i-1} \sum_{m=j+1}^n {\pi}_{km}
\end{align}
Of course, this bound is only relevant if the right hand side is greater than zero.
Taken together,~${\pi}_{ij}$ is bounded by,
\begin{align}
\ell_{ij} &\leq
{\pi}_{ij}
\leq
u_{ij}
\\
\ell_{ij} &\triangleq \max \left \{0, \, 1 - n + j - \sum_{k=1}^{j-1} {\pi}_{ik}  +  \sum_{k=1}^{i-1} \sum_{m=j+1}^n {\pi}_{km} \right \}
\\
u_{ij} &\triangleq 
\min \left \{1- \sum_{k=1}^{j-1} {\pi}_{ik}, \,
1- \sum_{k=1}^{i-1} {\pi}_{kj} \right\}.
\end{align}
Thus, we define,
\begin{align}
  {\pi}_{ij} &= \ell_{ij} + \sigma(\psi_{ij}) (u_{ij} - \ell_{ij}).
\end{align}

The inverse transformation from~${\Pi}$ to $\Psi$ is analogous.
We start by computing~$\psi_{11}$ and then progressively compute
upper and lower bounds and set,
\begin{align}
\psi_{ij} &= \sigma^{-1} \left( \frac{{\pi}_{ij} - \ell_{ij}}{u_{ij} - \ell_{ij}} \right ).
\end{align}


Notice that these bounds only depend on values of~${\Pi}$ that
have already been computed; i.e., those that are above or to the left of
the~$(i,j)$-th entry. Thus, the transformation from~$\Psi$ to~${\Pi}$
is feed-forward according to this ordering.  Consequently, the
Jacobian of the inverse transformation,~$\mathrm{d}\Psi / \mathrm{d} \Pi$,
is lower triangular, and its determinant is the product of its diagonal,
\begin{align}
\left| \frac{\mathrm{d} \Psi } {\mathrm{d} \Pi} \right|
&= \prod_{i=1}^{n-1} \prod_{j=1}^{n-1} \frac{\partial \psi_{ij} }{\partial {\pi}_{ij}} \\
&= \prod_{i=1}^{n-1} \prod_{j=1}^{n-1} \frac{\partial}{\partial {\pi}_{ij}}
\sigma^{-1} \left( \frac{{\pi}_{ij} - \ell_{ij}}{u_{ij} - \ell_{ij}} \right ) \\
&= \prod_{i=1}^{n-1} \prod_{j=1}^{n-1}
\left( \frac{1}{u_{ij} - \ell_{ij}} \right )
\left( \frac{u_{ij} - \ell_{ij}}{{\pi}_{ij} - \ell_{ij}} \right )
\left( \frac{u_{ij} - \ell_{ij}}{u_{ij} - {\pi}_{ij}} \right ) \\
&= \prod_{i=1}^{n-1} \prod_{j=1}^{n-1}
\frac{u_{ij} - \ell_{ij}}{({\pi}_{ij} - \ell_{ij}) (u_{ij} - {\pi}_{ij})}
\end{align}

With these two ingredients, we can write the density of~${\Pi}$,
\begin{align}
  \text{vec} (\Psi) &\sim \distNormal(\mu, \diag(\eta^2))
  \\
  {\Pi} &= f(\Psi) \\
  \implies
  p(\Pi \mid \mu, \diag(\eta^2)) &= \left|\frac{\mathrm{d} \Psi }{\mathrm{d} {\Pi}} \right|
  \distNormal(f^{-1}({\Pi}) \mid \mu, \diag(\eta^2))
\end{align}

Given the density and a differentiable mapping we can perform
variational inference with stochastic optimization of the ELBO.
We define a distribution over doubly stochastic matrices as a
reparameterization of a multivariate Gaussian distribution
over~$\Psi$. We can estimate gradients via the reparameterization
trick.

It is important to note that the transformation is only piecewise
continuous: the function is not differentiable at the points where
the bounds change; for example, when changing~$\Psi$ causes the
active upper bound to switch from the row to the column constraint
or vice versa.  I think we can argue that these discontinuities
will not have a severe effect on our stochastic gradient algorithm.

\bibliography{refs}
\bibliographystyle{icml2016}

\appendix

\section{Analytical Gradients}
What is~$\partial \pi_{ij} / \partial \psi_{kl}$?

\begin{itemize}
\item If~$(i,j) \prec (k,l)$, then~$\psi_{k,l}$ does not influence~$\pi_{i,j}$ and the partial derivative is zero.

\item If~$(i,j) = (k,l)$, then
  \begin{align}
    \frac{\partial \pi_{i,j}}{\partial \psi_{i,j}}
    &= \frac{\partial}{\partial \psi_{ij}} \left\{ \ell_{ij} + \sigma(\psi_{ij}) ( u_{ij} - \ell_{ij}) \right\} \\
    &= \sigma(\psi_{ij}) \sigma(-\psi_{ij}) (u_{ij} - \ell_{ij}) 
  \end{align}

\item If~$(i,j) \succ (k,l)$ then the~$\psi_{k,l}$ influences~$\pi_{i,j}$ through the bounds. That is, we need to compute:
  \begin{align}
    \frac{\partial \pi_{i,j}}{\partial \psi_{k,l}}
    &=
    \frac{\partial \pi_{i,j}}{\partial \ell_{i,j}} \frac{\partial \ell_{i,j}}{\partial \psi_{k,j}}
    +
    \frac{\partial \pi_{i,j}}{\partial u_{i,j}} \frac{\partial u_{i,j}}{\partial \psi_{k,j}} \\
    &= 
    \sigma(-\psi_{ij}) \frac{\partial \ell_{i,j}}{\partial \psi_{k,j}}
    +
    \sigma(\psi_{ij}) \frac{\partial u_{i,j}}{\partial \psi_{k,j}}
  \end{align}

  The upper bounds are easier. There are two cases:
  \begin{enumerate}
  \item If~$1-\sum_{k=1}^{i-1} \pi_{k,j} < 1-\sum_{l=1}^{j-1} \pi_{il}$, then the column sum is tight and the partial derivative is only non-zero for~$psi_{kl}$ such that~$l=j$. In this case, 
    \begin{align}
    \frac{\partial u_{i,j}}{\partial \psi_{k,j}}
    &= - \frac{\partial \pi_{k,j}}{\partial \psi_{k,j}}
    \end{align}
  \end{enumerate}
\end{itemize}

\end{document}
