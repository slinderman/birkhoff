% !TEX program = pdflatex
% FONTS
\usepackage[T1]{fontenc}

\usepackage{tgtermes}
\usepackage{amsmath}
\usepackage{amssymb}
%\usepackage[subscriptcorrection,
%            amssymbols,
%            mtpbb,
%            mtpcal,
%            nofontinfo  % suppresses all warnings
%           ]{mtpro2}
\usepackage{scalefnt,letltxmacro}
\LetLtxMacro{\oldtextsc}{\textsc}
\renewcommand{\textsc}[1]{\oldtextsc{\scalefont{1.10}#1}}
\usepackage[scaled=0.92]{PTSans}
\usepackage{inconsolata}

% COLOR
\usepackage[usenames,dvipsnames]{xcolor}
\definecolor{shadecolor}{gray}{0.9}

% SPACING and TEXT
\usepackage[final, expansion=alltext]{microtype}
\usepackage[english]{babel}
\usepackage{afterpage}
\usepackage{framed}
\usepackage{nicefrac}

% Define a paragraph header function
\DeclareRobustCommand{\parhead}[1]{\textbf{#1}~}

% paragraph helper
\DeclareRobustCommand{\PP}{\textcolor{Plum}{\texttt{\P}}~}
\DeclareRobustCommand{\pp}{\textcolor{Plum}{\texttt{\P}}~}

% COUNTERS
\renewcommand{\labelenumi}{\color{black!67}{\arabic{enumi}.}}
\renewcommand{\labelenumii}{{\color{black!67}(\alph{enumii})}}
\renewcommand{\labelitemi}{{\color{black!67}\textbullet}}

% FIGURES
\usepackage{graphicx}
\usepackage[labelfont=bf]{caption}
\usepackage[format=hang]{subcaption}

% TABLES
\usepackage{booktabs}

% BIBLIOGRAPHY
\usepackage{natbib}

% ALGORITHMS
\usepackage[algoruled]{algorithm2e}
\usepackage{listings}
\usepackage{fancyvrb}
\fvset{fontsize=\normalsize}

% HYPERREF
\usepackage[colorlinks, linktoc=all]{hyperref}
\usepackage[all]{hypcap}
\hypersetup{citecolor=Violet}
\hypersetup{linkcolor=black}
\hypersetup{urlcolor=MidnightBlue}

% CLEVEREF must come after HYPERREF
\usepackage[nameinlink]{cleveref}

% ACRONYMS
\usepackage[acronym,smallcaps,nowarn]{glossaries}
% \makeglossaries

% COLOR DEFINITIONS
\newcommand{\red}[1]{\textcolor{BrickRed}{#1}}
\newcommand{\orange}[1]{\textcolor{BurntOrange}{#1}}
\newcommand{\green}[1]{\textcolor{OliveGreen}{#1}}
\newcommand{\blue}[1]{\textcolor{MidnightBlue}{#1}}
\newcommand{\gray}[1]{\textcolor{black!60}{#1}}

% LISTINGS DEFINTIONS
\usepackage{listings}
\lstdefinestyle{mystyle}{
    commentstyle=\color{OliveGreen},
    numberstyle=\tiny\color{black!60},
    stringstyle=\color{BrickRed},
    basicstyle=\ttfamily\scriptsize,
    breakatwhitespace=false,
    breaklines=true,
    captionpos=b,
    keepspaces=true,
    numbers=none,
    numbersep=5pt,
    showspaces=false,
    showstringspaces=false,
    showtabs=false,
    tabsize=2
}
\lstset{style=mystyle}

\input{preamble/preamble_math.tex}

% \linenumbers

\title{Marginal Matching for the \\ Neural ``Permutation Problem''}
\author{Scott W Linderman}
\date{October 17, 2016}

\begin{document}
\maketitle

\section{Introduction}
\label{sec:introduction}

The nematode \textit{C. Elegans} is a unique model organism for
neuroscientists in that its connectome, or neural wiring diagram,
has been known for at least three decades \citep{white1986structure}.
Despite this knowledge, an understanding of the functional significance
of these synaptic connections has remained elusive.
Recently, \citet{kato2015global} measured the simultaneous activity of hundreds of
neurons in head-fixed \textit{C. Elegans}, providing an exciting
opportunity to tackle this problem.  With this data, we can attempt to infer
how the one neuron influences its post-synaptic neighbors
based on patterns of pre- and post-synaptic activity. However,
these attempts are stifled by a major obstacle: the identity of the
neurons, and hence the set of synaptic partners, is only partially
known. Before we can infer the functional weights of the connections,
we must first infer the neuron identities. To tackle this problem,
we propose to find a labeling of neurons that approximately
maximizes the marginal likelihood of the data, integrating over possible
values of the functional connection weights. 

\section{Problem Statement}
The adult hermaphrodite \textit{C. Elegans} has~${N=302}$ neurons, each of
which is given a unique name, like \texttt{AAVL} or \texttt{BWM}.
The ``connectome'' corresponds to a list of known synaptic connections
that are present in every individual, which we may represent as a
binary adjacency matrix,~${\bA \in \{0,1\}^{N \times N}}$,
where the entry~$a_{n,n'} \triangleq a_{n \to n'}$ is 1 if there exists a directed
synapse from neuron~$n$ to neuron~$n'$ and 0 otherwise. Since the
synapses in~\textit{C. Elegans} are all gap junctions, these
connections are undirected, making~$\bA$ symmetric. 

While the connectome has been known for decades \citep{white1986structure},
the \emph{weight}, or functional signficance, of those connections
remains largely a mystery. We will represent the collection of
functional weights as a matrix~${\bW \in \reals^{N \times N}}$,
where~$w_{n,n'} \triangleq w_{n \to n'}$ denotes the strength
of the connection from neuron~$n$ to neuron~$n'$. First, note
that this matrix is not necessarily symmetric since the influence
of a gap junction depends on the input resistance of the pre-
and post-synaptic neurons. Second, note that this is overparameterized:
the weighted adjacency matrix is actually given by the elementwise
product~$\bA \odot \bW$, rendering the values of~$w_{n \to n'}$
meaningless where~$a_{n \to n'}=0$. Nevertheless, this parameterization
will prove notationally convenient.

Our goal is to infer~$\bW$ from measurements of neural activity. 
First, consider the setting in which we record from a single worm
and observe the matrix~$\bY \in \reals^{T \times M}$,
where~$T$ is the number of time bins and~$M \leq N$ is the number of
observed neurons. For example, the entry~$y_{t,m}$ may denote the relative change in
fluorescence, i.e. the~${\Delta F/F}$, of neuron~$m$ at time~$t$.

We map these~$M$ neurons onto the complete list of~$N$ neurons
via a partial permutation,
\begin{align*}
\pi: \{1, \ldots, M\} \to \{1, \ldots, N\},
\end{align*}
where~$\pi$ is an injective mapping from observed neurons to known
neuron identities. In other words, a neuron identity can be assigned to at most one of
the~$M$ observed neurons.
We then define the partial permutation matrix,~$\bP \in \{0,1\}^{M \times N}$,
where~${P_{m,n} = \bbI[\pi(m)=n]}$, and~$\bbI[\cdot]$ is an indicator
function that evaluates to one if its argument is true and to zero otherwise.
Thus,~$\bP \bA \bP^\trans$ is the~$M \times M$ submatrix
of~$\bA$ corresponding to the~$M$ observed neurons.

In general, the mapping~$\pi$ is not known \emph{a priori}. We are typically
only certain about the labels of, say, twenty or thirty of the neurons.
Thus, to infer~$\bW$, we also need to infer~$\pi$. The next section
derives the marginal probability of a given permutation, a
natural objective for empirical Bayesian optimization.

\section{Marginal Probability of a Permutation}
We begin by formulating a prior distribution on both~$\bW$ and~$\pi$
(which determines~$\bP$).
The former will be taken to be a weak Gaussian distribution,
\begin{align*}
 \bw_n \sim \distNormal(\bmu, \bSigma), 
\end{align*}
where~$\bw_n$ is the~$n$-th column of~$\bW$. For example, we could
set~$\bmu = \boldsymbol{0}$ and~$\bSigma = \sigma^2 \bI$.
We do not specify the prior on~$\pi$ in detail here, but note that
it should depend on known and measured locations of the cells.

Next we specify an autoregressive model for the observed activity,
\begin{align*}
%  y_{t,m} &= \bx_t^\trans \bP (\bW \odot \bA) \bp_m^\trans  + \epsilon_{t,m}, \\
%          &= \bx_t^\trans \bP \bW \bp_m \odot \bx_t^\trans \bP \bA \bp_m^\trans  + \epsilon_{t,m}, \\
  y_{t,m} &= \bx_t^\trans \bP (\bw_{\pi(m)} \odot \ba_{\pi(m)})  + \epsilon_{t,m}, \\
  \epsilon_{t,m} &\sim \distNormal(0, \eta_{\pi(m)}^2),
\end{align*}
where~$\bw_{\pi(m)} \in \reals^N$ and~$\ba_{\pi(m)} \in \{0,1\}^N$ are the~$\pi(m)$-th
columns of~$\bW$ and~$\bA$, respectively,
and~$\bx_t \in \reals^M$ is a deterministic function of the
preceding activity,
\begin{align*}
  x_{t,m} &= f(y_{1,m}, \ldots, y_{t-1,m}).
\end{align*}
Note that this model ignores the activity of the~$(N-M)$ unobserved
neurons. 

Now we can write the joint probability of the model,
\begin{align*}
  p(\bY, \bW, \pi \given \bA) &=
    p(\bW) \, p(\pi) \prod_{t=1}^T \prod_{m=1}^M p(y_{t,m} \given \bx_t, \bW, \pi, \bA).
\end{align*}
Since both the prior on~$\bw_n$ and the likelihood for~$y_{t,m}$ are
linear and Gaussian in~$\bw_n$, the conditional distribution of~$\bw_n$
is Gaussian as well. Let~$\bX \in \reals^{T \times M}$ be a matrix with
rows given by~$\bx_t^\trans$. Fixing~$\pi$, we have,
\begin{align*}
  p(\bw_{\pi(m)} \given \bY, \pi, \bA) &\propto \distNormal(\bw_{\pi(m)} \given \bmu, \bSigma) \prod_{t=1}^T \distNormal(y_{t,m} \given \bx_t^\trans \bP (\bw_{\pi(m)} \odot \ba_{\pi(m)}), \eta_{\pi(m)}^2) \\
  &\propto \distNormal(\bw_{\pi(m)} \given \widetilde{\bmu}_{\pi(m)}, \widetilde{\bSigma}_{\pi(m)}),
\end{align*}
where
\begin{align*}
  \widetilde{\bSigma}_{\pi(m)} &= \left[ \bSigma^{-1} + \eta_{\pi(m)}^{-2} (\bP^\trans \bX^\trans \bX \bP) \odot (\ba_{\pi(m)} \ba_{\pi(m)}^\trans ) \right]^{-1}, \\
  \widetilde{\bmu}_{\pi(m)} &= \widetilde{\bSigma}_{\pi(m)} \left[\bSigma^{-1} \bmu + \eta_{\pi(m)}^{-2} (\bP^\trans \bX^\trans \by_{:,m}) \odot \ba_{\pi(m)} \right].
\end{align*}

Moreover, since the model is conditionally conjugate,
we can evaluate the \emph{marginal} probability of the observed data
and a permutation, integrating over the corresponding weights of the
network. We have,
\begin{align*}
  p(\bY, \pi \given \bA) &= \int p(\bY, \bW, \pi \given \bA) \, \mathrm{d}\bW \\
  &\propto p(\pi) \prod_{m=1}^M \int p(\by_m \given \bw_{\pi(m)}, \ba_{\pi(m)}) \, p(\bw_{\pi(m)})  \, \mathrm{d}\bW \\
  &= p(\pi) \prod_{m=1}^M \frac{ \left| \bSigma \right|^{-1/2} \exp \left\{ -\frac{1}{2} \bmu^\trans \bSigma^{-1} \bmu \right\}}
  { \left| \widetilde{\bSigma}_{\pi(m)} \right|^{-1/2} \exp \left\{ -\frac{1}{2} \widetilde{\bmu}_{\pi(m)}^\trans \widetilde{\bSigma}_{\pi(m)}^{-1} \widetilde{\bmu}_{\pi(m)} \right\}} \\
  &= p(\pi) \prod_{m=1}^M p(\by_m \given \pi, \bA).
\end{align*}

This provides a ``score'' that we can attempt to optimize over the
space of permutations. In the recordings of \citet{kato2015global},
a subset of~$L < M$ neurons have already been labeled, making the
problem slightly easier. Still, the number of such permutations is~$(M-L)!$,
and, to our knowledge, this poses an intractable optimization problem.
Instead, we propose a simplified version of this problem that can
be solved exactly.

\section{An Approximate Solution via Weighted Bipartite Matching}
We will attempt to find a permutation by solving a relaxed version
of the maximum marginal likelihood problem above. Specifically,
we will formulate the permutation inference problem as a
weighted bipartite matching problem, for which the Hungarian
algorithm \citep{kuhn1955hungarian} provides an exact solution
in cubic time.

We construct a bipartite graph with~$M-L$ nodes on the left
and~$N-L$ nodes on the right. The left nodes correspond to the
neurons for which we seek labels, and the right nodes correspond
to the possible neuron identities. For each pair of nodes~$m$ on the left
and~$n$ on the right, connect them with an edge of weight,~$e_{m,n}$.
The Hungarian algorithm provides a matching (a subset of~$M-L$ edges such that
each left node is connected to exactly one right node and each right
node has at most one edge) such that the sum of edge weights is
maximized.  Furthermore, it does so in~$O((M-L)^2 (N-L)^2)$ time.
The question is, how should we set the edge weights?

It is not clear how to set the edge weights to correspond to the
objective function above since the marginal probability depends
on an entire assignment,~$\pi(1), \ldots, \pi(M)$. For the matching
problem, we want an objective function that just depends on a single
assignment,~$\pi(m)$. A simple approximation is to set,
\begin{align*}
  e_{m,n} &= p(\pi(m)=n) \, p(\by_m \given \pi(m)=n, \widetilde{\bA}(m,n)) \prod_{\ell = 1}^L p(\by_\ell \given \pi(m)=n, \widetilde{\bA}(m,n)),
\end{align*}
excusing the abuse of notation.
Here, we assume the prior,~$p(\pi)$ factorizes such that we can evaluate
the marginal,~$p(\pi(m)=n)$. There are two approximations:
\begin{enumerate}
\item We only consider the marginal probability of only the \emph{labeled}
neurons,~$\ell=1, \ldots, L$, and one unlabeled neuron,~$m$.

\item We define a ``pseudo-adjacency matrix''~$\widetilde{\bA}(m,n)$
that zeros out all entries except for the~$(L+1) \times (L+1)$ submatrix
corresponding to the labeled neurons and observed neuron~$m$, which we
are assigning to label~$\pi(m)=n$. More formally, we define~$\widetilde{\bA}(m,n)$
with entries~$\widetilde{a}_{i \to j}(m,n)$ as follows,
\begin{align*}
  \widetilde{a}_{i \to j}(m,n) &=
  \begin{cases}
    a_{i \to j}  & \text{both $i$ and~$j$ are either labeled or equal to~$n$} \\ 
    0 & \text{o.w.}
  \end{cases}
\end{align*}
\end{enumerate}

This pseudo-adjacency matrix leads to the following interpretation of the edge
weight:~$e_{m,n}$ is the marginal probability of the activity of the labeled neurons
after including a single extra neuron,~$m$, with label~$\pi(m)=n$. While this
ignores a significant amount of information (i.e. the~$M-L-1$ other neurons, it
successfully renders the edge weights independent of one another. Alternatively,
we could set the other edges of~$\widetilde{\bA}(m,n)$ to one rather than
zero in order to include all other possible interactions, even though many
would truly be zero.

\section{A Greedy Approach}
For this problem, the~$O(M^2 N^2)$ complexity of the Hungarian algorithm
may be unacceptable. A simpler approach is to just do a greedy fit. Using
the same edge weights as above, we could iterate over each unlabeled
neuron,~$m$, and each possible assignment,~$n$, evaluate the weight
of the corresponding edges, and add the assignment~$\pi(m)=n$ for the
highest weighted edge. After making this assignment, we add~$m$ to the
set of labeled neurons and repeat. The complexity of this algorithm
is only~$O(M^2 N)$. However, note that evaluating the edge weights
is, in the worst case,~$O(M^3)$ as well. Yikes! 

\bibliographystyle{apa}
\bibliography{refs.bib}

\end{document}






































